%! TEX root = paper.tex

\section{Оценка эффективности системы информационной безопасности предприятия}
\label{sec:efficiency}

Создаваемая система информационной защиты должна быть
эффективна с технической точки зрения, но также при её разработке
необходимо учитывать экономическую сторону вопроса. Принять точное
решение о необходимых действиях для уменьшения риска довольно сложно,
поскольку в разных ситуациях существуют различные вероятности угроз.
Существует корреляция между уменьшением количества уязвимостей и
стоимостью подобной защиты.

При проектировании системы следует провести оценку эффективности. Обычно адекватность мероприятий по защите информации оценивается методами анализа рисков, которые изложены в стандартах серии ISO 13335. Существует еще один подход при оценке защиты информации -- это расчет метрик, при котором отдельные аспекты безопасности информационной системы характеризуются одним или несколькими численными показателями. Также существует тестирование на проникновение, при котором
демонстрируются атаки и на практике выявляются слабые места системы.

Стоит отметить, что соблюдение оговоренных требований является
достаточной мерой для оценки эффективности управления. Только в этом
случае уровень эффективности управления в соответствии со всеми
стандартами можно будет считать удовлетворительным \cite{medium_cybersecurity_kpi}.

