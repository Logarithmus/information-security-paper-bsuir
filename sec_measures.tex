%! TEX root = paper.tex

\section{Мероприятия по реализации мер информационной безопасности предприятия}
\label{sec:measures}


\subsection{Организационное обеспечение информационной безопасности}
\label{subsec:measures:organisational}

Эффективная защита информации возможна при обязательном выполнении ряда условий:
\begin{itemize}
	\item единство в решении производственных, коммерческих,
финансовых и режимных вопросов;
	\item координация мер безопасности между всеми заинтересованными подразделениями предприятия;
	\item научная оценка информации и объектов, подлежащих классификации (защите), а также разработка режимных мер до начала проведения режимных работ;
	\item персональная ответственность (в том числе и материальная) руководителей всех уровней, исполнителей, участвующих в закрытых работах, за обеспечение сохранности тайны и поддержание на должном уровне режима охраны проводимых работ;
	\item включение основных обязанностей рабочих, специалистов и администрации по соблюдению конкретных требований режима в коллективный договор, контракт, трудовое соглашение, правила трудового распорядка;
	\item организация специального делопроизводства, порядка хранения, перевозки носителей тайны;
	\item введение соответствующей маркировки документов и других носителей закрытых сведений;
	\item формирование списка лиц, уполномоченных руководителем предприятия классифицировать информацию и объекты, содержащие конфиденциальные сведения;
	\item ограничение числа лиц, допускаемых к защищаемой информации;
	\item наличие единого порядка доступа и оформления пропусков;
	\item выполнение требований по обеспечению сохранения защищаемой информации при проектировании и размещении специальных помещений, в процессе опытно-конструкторской разработки, испытаний и производства изделий, сбыта, рекламы, подписания контрактов, при проведении особо важных совещаний, в ходе использования технических средств обработки, хранения и передачи информации и т. п.;
	\item организация взаимодействия с государственными органами власти, имеющими полномочия по контролю определенных видов деятельности предприятий и фирм;
	\item наличие охраны, пропускного и внутриобъектового режимов;
	\item плановость разработки и осуществления мер по защите информации, систематический контроль за эффективностью принимаемых мер;
	\item создание системы обучения исполнителей правилам обеспечения сохранности информации.
\end{itemize}


\subsubsection{}
\label{subsubsec:measures:organisational:tasks}

\textit{Задачи организационного обеспечения информационной безопасности.} Для соблюдения условий эффективной защиты информации требуется выполнение следующих задач:
\begin{itemize}
	\item организация специального делопроизводства, порядка хранения, перевозки носителей тайны;
	\item введение соответствующей маркировки документов и других носителей закрытых сведений;
	\item формирование списка лиц, уполномоченных руководителем предприятия классифицировать информацию и объекты, содержащие конфиденциальные сведения;
	\item оптимальное ограничение числа лиц, допускаемых к защищаемой информации;
	\item наличие единого порядка доступа и оформления пропусков;
	\item выполнение требований по обеспечению сохранения защищаемой информации при проектировании и размещении специальных помещений, в процессе опытно-конструкторской разработки, испытаний и производства изделий, сбыта, рекламы, подписания контрактов, при проведении особо важных совещаний, в ходе использования технических средств обработки, хранения и передачи информации и т. п.;
	\item организация взаимодействия с государственными органами власти, имеющими полномочия по контролю определенных видов деятельности предприятий и фирм;
	\item наличие охраны, пропускного и внутриобъектного режимов;
	\item плановость разработки и осуществления мер по защите информации, систематический контроль за эффективностью принимаемых мер;
	\item создание системы обучения исполнителей правилам обеспечения сохранности информации \cite{vostrezova_ural}.
\end{itemize}

\subsubsection{}
\label{subsubsec:measures:organisational:units}

\textit{Подразделения, занятые в обеспечении информационной безопасности.} Руководители предприятий, организаций, учреждений в соответствии со своими должностными обязанностями при деятельности, связанной с информацией, которая составляет государственную или иную тайну, создают службу (подразделение) по защите информации.
Для организации соответствующей деятельности они издают нормативные правовые акты (приказы, распоряжения), а также утверждают руководства, инструкции, положения, правила, методические рекомендации, касающиеся защиты информации и деятельности служб защиты информации.
Для деятельности, связанной с государственной тайной, предприятие должно иметь лицензию на этот вид деятельности, все средства защиты должны быть сертифицированы \cite{vostrezova_ural}.

Определение потребности службы ИБ предприятия в кадровых ресурсах заключается в установлении необходимого количества, а также требуемой компетенции работников службы ИБ, выполняемой на основе:
\begin{itemize}
	\item анализа задач и функций, возложенных на службу ИБ предприятия;
	\item уровня автоматизации процессов службы обеспечения ИБ и централизации управления средствами автоматизации;
	\item прогноза возможного расширения состава задач и функций службы ИБ в соответствии с планами совершенствования процессов службы вследствие развития бизнес-процессов предприятия.
\end{itemize}

Служба ИБ должна иметь утверждённые руководством полномочия и ресурсы, необходимые для выполнения установленных целей и задач, а также назначенного из числа руководства куратора. Кроме того, рекомендуется наделить службу ИБ собственным бюджетом \cite{nbrb}.

\subsubsection{}
\label{subsubsec:measures:organisational:intercommunication}

\textit{Взаимодействие подразделений, занятых в обеспечении информационной безопасности} должно происходить на основании внутренних правил, инструкций, руководств и положений, принятых на предприятии.

\subsection{Техническое обеспечение информационной безопасности предприятия}
\label{subsec:measures:technical}

\subsubsection{}
\label{subsubsec:measures:techical:general}

\textit{Общие положения.}
Технологии защиты информации можно разделить на три группы:
\begin{itemize}
	\item программные;
	\item сетевые;
	\item аппаратные.
\end{itemize}

\subsubsection{}
\label{subsubsec:measures:techical:protection}

\textit{Защита информационных ресурсов от несанкционированного доступа.} Существует несколько сетевых технологий, используемых для защиты ресурсов организации.

Виртуальные частные сети (VPN) являются защищенными виртуальными сетями, созданными на основе общедоступной сети (например, Интернет). Безопасность VPN обеспечивается шифрованием содержимого пакетов, передаваемых между устройствами, входящими в эту сеть.

Контроль доступа к сети (NAC) выполняет ряд проверок, прежде чем разрешить устройству подключение к сетевой инфраструктуре. Стандартные проверки включают установку обновлений антивирусного ПО и обновлений операционной системы.

Обеспечение безопасности точки беспроводного доступа включает в себя процессы аутентификации и шифрования \cite{cisco_network_security}.

\subsubsection{}
\label{subsubsec:measures:techical:comprehensive}

\textit{Средства комплексной защиты от потенциальных угроз.} Программные меры защиты включают программы и сервисы, которые защищают операционные системы, базы данных и другие сервисы на рабочих станциях, портативных устройствах и серверах.
Администраторы устанавливают программные меры защиты или контрмеры на отдельных хостах или серверах.

Программные межсетевые экраны контролируют удаленный доступ к системе. Операционные системы, как правило, имеют встроенный межсетевой экран. В качестве альтернативы пользователь может приобрести или загрузить программное обеспечение от сторонних производителей.

Сканеры сетевой инфраструктуры и портов обнаруживают и контролируют открытые порты на хосте или сервере.

Анализаторы протоколов или сигнатурные анализаторы являются устройствами, которые собирают и анализируют сетевой трафик. Они выявляют проблемы производительности, некорректные настройки, сбои приложений, устанавливают базовые и нормальные показатели трафика и позволяют отладить проблемы связи.

Сканеры уязвимостей -- это компьютерные программы, предназначенные для оценки слабых мест компьютеров или сетевых инфраструктур.

Системы обнаружения вторжений (IDS) на основе хоста ведут мониторинг событий только на хостах. При выявлении необычных событий IDS генерирует файлы журналов и уведомления. Система, которая хранит конфиденциальные данные или предоставляет критически важные сервисы, является кандидатом для систем обнаружения вторжений на основе хоста \cite{cisco_software_security}.

\subsubsection{}
\label{subsubsec:measures:techical:qa}

\textit{Обеспечение качества в системе безопасности} играет важную роль.
За обеспечением исполнения норм и правил ИБ должна следить служба ИБ предприятия.
Основным критерием качества в ИБ является устойчивость информационных ресурсов организации к возможным вторжениям со стороны злоумышленников и неопытных пользователей.

\subsubsection{}
\label{subsubsec:measures:technical:service_staff}

\textit{Принципы организации работ обслуживающего персонала} предприятия должны устанавливаться соответствующими правилами и нормами.
Данные правила и нормы разрабатываются службой ИБ и утверждаются руководством предприятия.

\subsection{Правовое обеспечение информационной безопасности предприятия}
\label{subsec:measures:law}

\subsubsection{}
\label{subsubsec:measures:law:employees}
\textit{Правовое обеспечение юридических отношений с работниками предприятия}.
Трудовой кодекс Республики Беларусь содержит нормы, согласно которым работники организаций
обязаны хранить государственную и служебную тайну, не разглашать
коммерческую тайну работодателя, коммерческую тайну третьих лиц, к
которой работодатель получил доступ (п.10 части 1 статьи 53) \cite{labor_kodeks}.

Кодекс Республики Беларусь об административных правонарушениях устанавливает административные и правовые санкции за правонарушения в информационной сфере.
К таким правонарушениям относятся:
\begin{itemize}
	\item отказ предоставить гражданину информацию, которая косвенно затрагивает его
права, свободы и законные интересы (статья 9.6);
	\item несанкционированный доступ к информации (статья 22.6);
	\item нарушение правил защиты информации (статья 22.7) и др. \cite{koap_rb}
\end{itemize}

\subsubsection{}
\label{subsubsec:measures:law:partners}

\textit{Правовое обеспечение юридических отношений с партнерами предприятия.}
Гражданский кодекс Республики Беларусь содержит положения об
официальной и коммерческой тайне, устанавливает такие формы отношений,
как информационные услуги,  сделка предусматривает ответственность
за неправомерное использование информации (статья 140, часть 2 статьи 161,
статья 1011 и др.) \cite{civil_kodeks}.

\subsubsection{}
\label{subsubsec:measures:law:digital_signature}

\textit{Правовое обеспечение применения электронной цифровой подписи.}
В Республике Беларусь электронная подпись признаётся как средство подтверждения подлинности соглашения сторон (Закон Республики Беларусь <<Об электронном документе и электронной цифровой подписи>> от 28 декабря 2009 г. № 113-З) \cite{law_ecp}.
