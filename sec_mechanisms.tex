%! TEX root = paper.tex

\section{Механизмы обеспечения информационной безопасности предприятия}
\label{sec:mechanisms}

\subsection{Принципы, условия и требования к организации и функционированию системы информационной безопасности}
\label{subsec:mechanisms:principles}

При создании системы информационной безопасности (далее ИБ)
необходимо придерживаться определённых принципов \cite{vasilkov}:

Конфиденциальность. Это свойство, при котором информация не
предоставляется или не раскрывается неавторизованным лицам, организациям
или процессам. Доступ к конфиденциальной информации, ее использование,
копирование или раскрытие разрешается только авторизованным
пользователям. Нарушение конфиденциальности происходит, если
посторонние лица или системы получают доступ или раскрывают
информацию, которая им не разрешена. Например, чтобы предотвратить
раскрытие конфиденциальных данных, таких как номер кредитной карты, от
перехватчиков, передача должна быть зашифрована. Кроме того, номер
должен быть защищен везде, где он будет обрабатываться или храниться
(например, в базах данных), чтобы предотвратить несанкционированный
доступ. Документы, хранящиеся в офисе нотариуса, должны быть защищены
от копирования и посторонних лиц. Также нотариус должен обеспечить
полную закрытость имеющейся у него информация. Нотариус обязан хранить
в тайне сведения, которые стали ему известны в связи с осуществлением его
профессиональной деятельности, поэтому утечки данных недопустимы.

Целостность. В информационной безопасности целостность
означает, что информация не может быть изменена или подделана без
обнаружения. Он обеспечивает правильность сообщения и защищает от
несанкционированного изменения. Если информация была изменена,
изменится и хэш-значение файла или код аутентификации сообщения (MAC)
сообщения. Таким образом, изменение будет распознаваться при сравнении с
исходной информацией.

Доступность. Предполагает, что информационные системы и
услуги, а также сама информация доступны и работают должным образом,
когда это необходимо или запрашивается. Его также можно рассматривать как
степень работоспособности системы или оборудования.

Подлинность (аутентификация). Доказывает, что все стороны, участвующие в действии, являются теми, кем они себя называют, путем подтверждения своей личности.

Безотказность. Это возможность доказать наличие заявленного
события или действия и его исходных сущностей, чтобы разрешить споры о
возникновении или ненаступлении события или действия и причастности
сущностей к событию. В информационных технологиях и коммуникациях
безотказность гарантирует, что отправителю данных предоставляется
доказательство доставки, а получателю – подтверждение личности
отправителя, поэтому никто не может впоследствии отрицать обработку
данных. Например, цифровые подписи используются для подтверждения
подлинности и предотвращения отказа от авторства.

Подотчетность (ответственность). Устанавливает ответственность
организации (например, человека) за свои действия и решения. С этой целью
все соответствующие события и операции в системе, например, неудачные и
успешные попытки аутентификации, записываются в журнал. Контрольный
журнал, также называемый информационным аудитом, представляет собой
хронологическую запись действий системы, позволяющую реконструировать
и исследовать последовательность событий.

Законность. Подразумевается, что проектирование системы ИБ не
нарушает действующего законодательства и нормативных актов. Предприятие
должно использовать только дозволенные методы обеспечения ИБ.

Простота и автоматизация. Механизмы защиты должны быть по
возможности интуитивны и не требовать от рядовых пользователей особых
знаний и избыточных действий.

\subsection{Основные направления политики в сфере информационной безопасности}
\label{subsec:mechanisms:directions}

Под термином «политика безопасности» понимается совокупность
правил, организационно-технических и режимных мер и методов,
определяющих и ограничивающих виды деятельности объектов и субъектов,
системы информационной безопасности. От степени проработанности
политик зависит вероятность инцидентов.
Политика безопасности распространяется на личные данные,
записанные на бумаге, а также записанные в информации и на носителях
данных.
Перечислим основные политики ниже:
\begin{itemize}
	\item политика управления доступом к ресурсам корпоративной сети;
	\item политика обеспечения безопасности удаленного доступа к ресурсам корпоративной сети организации;
	\item политика обеспечения безопасности при взаимодействии с глобальной международной сетью Интернет;
	\item политика и регламент резервного копирования и восстановления данных;
	\item парольная политика;
	\item антивирусная политика;
	\item политика обеспечения актуальности информации;
	\item политика обновления прикладных ресурсов;
	\item политика и регламент резервного копирования и восстановления
данных.
\end{itemize}

\subsection{Планирование мероприятий по обеспечению информационной безопасности предприятия}
\label{subsec:mechanisms:planning}

\subsection{Критерии и показатели информационной безопасности предприятия}
\label{subsec:mechanisms:indicators}

