%! TEX root = paper.tex

\section{Механизмы обеспечения информационной безопасности предприятия}
\label{sec:mechanisms}

\subsection{Принципы, условия и требования к организации и функционированию системы информационной безопасности}
\label{subsec:mechanisms:principles}

При создании системы информационной безопасности (далее ИБ)
необходимо придерживаться определённых принципов \cite{vasilkov}:

Конфиденциальность. Это свойство, при котором информация не
предоставляется или не раскрывается неавторизованным лицам, организациям
или процессам. Доступ к конфиденциальной информации, ее использование,
копирование или раскрытие разрешается только авторизованным
пользователям. Нарушение конфиденциальности происходит, если
посторонние лица или системы получают доступ или раскрывают
информацию, которая им не разрешена. Например, чтобы предотвратить
раскрытие конфиденциальных данных, таких как секретные чертежи или исходный код программы,
от перехватчиков, передача должна быть зашифрована.

Целостность. В информационной безопасности целостность
означает, что информация не может быть изменена или подделана без
обнаружения. Он обеспечивает правильность сообщения и защищает от
несанкционированного изменения. Если информация была изменена,
изменится и хэш-сумма сообщения.

Доступность. Предполагает, что информационные системы и
услуги, а также сама информация доступны и работают должным образом,
когда это необходимо или запрашивается. Его также можно рассматривать как
степень работоспособности системы или оборудования.

Подлинность (аутентификация). Доказывает, что все стороны, участвующие в действии, являются теми, кем они себя называют, путем подтверждения своей личности.

Безотказность. Это возможность доказать наличие заявленного
события или действия и его исходных сущностей, чтобы разрешить споры о
возникновении или ненаступлении события или действия и причастности
сущностей к событию. В информационных технологиях и коммуникациях
безотказность гарантирует, что отправителю данных предоставляется
доказательство доставки, а получателю – подтверждение личности
отправителя, поэтому никто не может впоследствии отрицать обработку
данных. Например, цифровые подписи используются для подтверждения
подлинности и предотвращения отказа от авторства.

Подотчетность (ответственность). Устанавливает ответственность
организации (например, человека) за свои действия и решения. С этой целью
все соответствующие события и операции в системе, например, неудачные и
успешные попытки аутентификации, записываются в журнал. Контрольный
журнал, также называемый информационным аудитом, представляет собой
хронологическую запись действий системы, позволяющую реконструировать
и исследовать последовательность событий.

Законность. Подразумевается, что проектирование системы ИБ не
нарушает действующего законодательства и нормативных актов. Предприятие
должно использовать только дозволенные методы обеспечения ИБ.

Простота и автоматизация. Механизмы защиты должны быть по
возможности интуитивны и не требовать от рядовых пользователей особых
знаний и избыточных действий.

\subsection{Основные направления политики в сфере информационной безопасности}
\label{subsec:mechanisms:directions}

Под термином «политика безопасности» понимается совокупность
правил, организационно-технических и режимных мер и методов,
определяющих и ограничивающих виды деятельности объектов и субъектов,
системы информационной безопасности. От степени проработанности
политик зависит вероятность инцидентов.
Политика безопасности распространяется на личные данные,
записанные на бумаге, а также записанные в информации и на носителях
данных.
Перечислим основные политики ниже:
\begin{itemize}
	\item политика управления доступом к ресурсам корпоративной сети;
	\item политика обеспечения безопасности удаленного доступа к ресурсам корпоративной сети организации;
	\item политика обеспечения безопасности при взаимодействии с глобальной международной сетью Интернет;
	\item политика резервного копирования и восстановления данных;
	\item парольная политика;
	\item антивирусная политика;
	\item политика обеспечения актуальности информации;
	\item политика обновления прикладных ресурсов.
\end{itemize}

\subsection{Планирование мероприятий по обеспечению информационной безопасности предприятия}
\label{subsec:mechanisms:planning}

В целях выполнения задач по обеспечению информационной безопасности научно-проектного предприятия, в соответствии с рекомендациями беларуских и международных стандартов по безопасности на предприятии должны быть определены следующие роли:
\begin{itemize}
	\item ответственное подразделение;
	\item работник научно-проектного предприятия.
\end{itemize}

При необходимости могут быть определены и другие роли по информационной безопасности.
Оперативная деятельность и планирование деятельности по обеспечению информационной безопасности научно-проектного предприятия осуществляются и координируются \textit{ответственным подразделением}.
В задачи \textit{ответственного подразделения} входит:
\begin{itemize}
	\item установление потребностей предприятия в применении мер обеспечения информационной безопасности, определяемых внутренними правилами и требованиями нормативно-правовых актов;
	\item разработка и пересмотр внутренних нормативных документов по обеспечению информационной безопасности предприятия, в т. ч. планы, политики, положения, регламенты, инструкции, методики, перечни сведений и иные виды внутренних нормативных документов;
	\item контроль актуальности внутренних документов и проверка их на соответствие законодательству;
	\item обучение, контроль и работа с персоналом предприятия в области обеспечения информационной безопасности;
	\item выявление и предотвращение реализации угроз ИБ;
	\item информирование ответственных лиц о вероятных угрозах ИБ;
	\item пресечение действий нарушителей ИБ \cite{bsmu_privacy}.
\end{itemize}

Основные задачи работников научно-проектного предприятия при выполнении должностных обязанностей и в рамках их участия в оперативной деятельности по обеспечению информационной безопасности:
\begin{itemize}
	\item осознание персональной ответственности за свои действия при нарушении правил ИБ;
	\item соблюдение требований ИБ, установленных на предприятии;
	\item выявление и реагирование на инциденты информационной безопасности в рамках компетенции;
	\item информирование в установленном порядке ответственных лиц о нарушениях ИБ;
	\item прогнозирование и предупреждение инцидентов ИБ в рамках компетенции;
	\item мониторинг и оценка ИБ в рамках своего участка работы (рабочего места, структурного подразделения) и в пределах своей компетенции \cite{bsmu_privacy}.
\end{itemize}

\subsection{Критерии и показатели информационной безопасности предприятия}
\label{subsec:mechanisms:indicators}

С развитием информационных технологий появилась необходимость стандартизации требований в области защиты информации.
Главная задача стандартов информационной безопасности -- создать основу для взаимодействия между производителями, потребителями и специалистами по сертификации.
Каждая из этих групп имеет свои интересы и свои взгляды на проблему информационной безопасности.

Производители нуждаются в стандартах как средстве сравнения возможностей своих продуктов, и в применении процедуры сертификации как механизме оценки их свойств, а также в стандартизации определенного набора требований безопасности, который мог бы ограничить фантазию заказчика конкретного продукта и заставить его выбирать требования из этого набора.

Потребители заинтересованы в методике, позволяющей обоснованно выбрать продукт, отвечающий их нуждам и решающий их проблемы, для чего им необходима шкала оценки безопасности и инструмент, с помощью которого они могли бы формулировать свои требования производителям.

Специалисты по сертификации рассматривают стандарты как инструмент, позволяющий им оценить уровень безопасности, обеспечиваемый системой, и предоставить потребителям возможность сделать обоснованный выбор. Специалисты по сертификации заинтересованы в четких и простых критериях, так как они должны дать обоснованный ответ пользователям -- удовлетворяет продукт их нужды, или нет.
В конечном счете именно они принимают на себя ответственность за безопасность продукта, получившего квалификацию уровня безопасности и прошедшего сертификацию.

Таким образом, перед стандартами информационной безопасности стоит непростая задача создать эффективный механизм взаимодействия всех сторон \cite{vostrezova_ural}.

\textit{Европейские критерии безопасности информационных технологий\\(ITSEC.}
Обзор основывается на версии 1.2 этих критериев, опубликованной в июне 1991 года от имени четырех стран: Франции, Германии, Нидерландов и Великобритании.
Европейские критерии рассматривают следующие задачи средств информационной безопасности:
\begin{itemize}
	\item защита информации от несанкционированного доступа;
	\item обеспечение целостности информации посредством защиты от ее несанкционированной модификации или уничтожения;
	\item обеспечение работоспособности систем с помощью противодействия угрозам отказа в обслуживании.
\end{itemize}

Иными словами, основные критерии ИБ можно сформулировать следующим образом:
\begin{itemize}
	\item конфиденциальность;
	\item целостность;
	\item доступность.
\end{itemize}

В \textit{ITSEC} проводится различие между системами и продуктами.

\textit{Система} -- это конкретная аппаратно-программная конфигурация, построенная с вполне определенными целями и функционирующая в известном окружении.

\textit{Продукт} -- это аппаратно-программный <<пакет>>, который можно купить и по своему усмотрению встроить в ту или иную систему.

Таким образом, с точки зрения информационной безопасности основное отличие между системой и продуктом состоит в том, что система имеет конкретное окружение, которое можно определить и изучить сколь угодно детально, а продукт должен быть рассчитан на использование в различных условиях.
Угрозы безопасности системы носят вполне конкретный и реальный характер. Относительно угроз продукту можно лишь строить предположения. Разработчик может специфицировать условия, пригодные для функционирования продукта; дело покупателя обеспечить выполнение этих условий.
