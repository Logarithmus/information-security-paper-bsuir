%! TEX root = paper.tex

\section{Проблемная ситуация в сфере информационной безопасности}
\label{sec:problems}

Объекты ИБ -- предметы и лица, подлежащие защите в рамках системы информационной безопасности.
Для научно-производственного предприятия можно привести следующие примеры объектов ИБ:
\begin{itemize}
	\item эскизы, концептуальные макеты, чертежи выпускаемой и разрабатываемой продукции (как на электронных, так и на бумажных носителях информации), представляющие интерес со стороны конкурентов;
	\item информационная инфраструктура, включающая технические, программные и аппаратно-программные средства обработки и передачи информации (например, оборудование локальной вычислительной сети, рабочие станции сотрудников, экспериментальные установки, станки с программным управлением, 3D – принтеры и др.) \cite{national_traffic_exchange}.
\end{itemize}

\subsection{Объекты информационной безопасности}
\label{subsec:problems:objects}

Нарушитель -- это лицо, предпринявшее попытку выполнения действий, запрещённых политикой ИБ по ошибке, незнанию либо осознанно со злым умыслом (из корыстных интересов).
Злоумышленник -- нарушитель, намеренно идущий на нарушение политики ИБ из корыстных побуждений.
Модель нарушителя отражает его теоретические и практические возможности, знания и место действия и др.
Составив реалистичную модель нарушителя для конкретной системы или предприятия, можно предотвратить возникновение причин совершения нарушения, а также точнее определить требования к системе защиты информационной системы.
При разработке модели нарушителя определяются:
\begin{itemize}
	\item предположения о категориях лиц, к которым может принадлежать потенциальный нарушитель;
	\item предположения о мотивах действий нарушителя;
	\item предположения о квалификации нарушителя и его технической оснащённости (об используемых методах и средствах);
	\item ограничения и предположения.
\end{itemize}

\subsection{Определение вероятного нарушителя}
\label{subsec:problems:determination}

Для заблаговременного прогнозирования вероятности возникновения различных угроз, связанных с информационной безопасностью, принципиально важным представляется использование различных методов.
Обнаружение вторжений –- это процесс выявления и реагирования на
злонамеренные действия в отношении информационных объектов.
Это процесс, который должен постоянно выполняться.
Злоумышленник может быть идентифицирован до, во время или после совершения вредоносной деятельности.
Предупреждающие меры могут защитить или частично сохранить ресурсы.
Возможная деятельность после совершения взлома обычно должна быть связана с оценкой ущерба от атаки и определением причин взлома.
Действия во время проникновения связаны с принятием решения, позволить ли проникновению продолжаться, чтобы наблюдать за нарушителем, или начать тревогу и, возможно, отпугнуть его.
Точное решение может быть только после идентификации нарушения. \cite{vostrezova_ural}

\subsection{Описание особенностей (профиля) каждой из групп вероятных нарушителей}
\label{subsec:problems:profile}

Нарушитель — это лицо, предпринявшее попытку выполнения запрещенных операций (действий) по ошибке, незнанию или осознанно со злым умыслом (из корыстных интересов) или без такового (ради игры или удовольствия, с целью самоутверждения и т. п.) и использующее для этого различные возможности, методы и средства.

Злоумышленник — нарушитель, намеренно идущий на нарушение из корыстных побуждений.
Профиль нарушителя отражает его практические и теоретические возможности, априорные знания, время и место действия и т. п.
Исследовав причины нарушений, можно либо повлиять на сами эти причины, либо точнее определить требования к системе защиты от данного вида нарушений или преступлений.
В каждом конкретном случае исходя из конкретной технологии обработки информации может быть определена модель нарушителя, которая должна быть адекватна реальному нарушителю для данной системы.
Модель нарушителя разрабатывается при проектировании системы защиты и оценке защищенности информации.
При разработке модели нарушителя определяются:
\begin{itemize}
	\item предположения о категориях лиц, к которым может принадлежать нарушитель;
	\item предположения о мотивах действий нарушителя (преследуемых нарушителем целях);
	\item предположения о квалификации нарушителя и его техческой оснащенности (об используемых для совершения нарушения методах и средствах);
	\item ограничения и предположения о характере возможных действий нарушителей.
\end{itemize}

По отношению к системе нарушители могут быть внутренними (из числа персонала системы) или внешними (посторонними лицами). Практика показывает, что на долю внутренних
нарушителей приходится более 2/3 от общего числа нарушений. \cite{vasilkov}

\subsection{Основные виды угроз информационной безопасности предприятия}
\label{subsec:problems:threats}

\subsubsection{}
\label{subsubsec:problems:threats:classification}

\textit{Классификация угроз}. Угрозы информационной безопасности можно примерно разделить по
следующим критериям:
\begin{itemize}
	\item случайные угрозы -- стихийные бедствия, катастрофы, несчастные
случаи, которые влияют на статус информационной безопасности
организации (например, пожар в здании, в котором хранятся носители
информации);
	\item традиционные информационные угрозы -- шпионская, подрывная или диверсионная деятельность (направленная на получение информации или на дезинформацию; совершается другими лицами и организациями);
	\item технологические угрозы -- угрозы, связанные со сбором, хранением и обработкой информации в сетях ИКТ (например, компьютерные преступления, кибертерроризм, информационная борьба);
	\item угрозы, связанные с гражданскими правами отдельных лиц или социальных групп (например, продажа информации или ее передача несанкционированным лицам, нарушение властями неприкосновенности частной жизни, незаконное вмешательство спецслужб, ограничение прозрачности \cite{devyanin} общественной жизни).
\end{itemize}

\subsubsection{}
\label{subsubsec:problems:threats:unintentional}

\textit{Основные непреднамеренные искусственные угрозы}.
К непреднамеренным искусственным угрозам можно отнести:
\begin{itemize}
	\item случайное инициирование сбоя неопытным пользователем;
	\item использование флеш-накопителей, заражённых вирусом, не подозревающим об этом человеком;
	\item загрузка и установка шпионского ПО, выдающего себя за полезное.
\end{itemize}

\subsubsection{}
\label{subsubsec:problems:threats:intentional}

\textit{Основные преднамеренные искусственные угрозы}. Преднамеренная угроза для системы информационной безопасности является результатом накопления трёх элементов: мотива, средства осуществления взлома системы и возможности получить доступ к диску компьютера и сети.
Человек может использовать различные методы взлома информационных систем, такие как:
\begin{itemize}
	\item преднамеренное инициирование сбоя;
	\item запуск ложных срабатываний (усыпление бдительности);
	\item шантаж, коррупция;
	\item взлом пароля доступа;
	\item прослушивание сети;
	\item использование вирусов, червей, троянов, логических бомб \cite{zhigulin} и других опасных приложений, дестабилизирующих эффективность системы;
	\item использование уязвимостей в доступе к электронной почте и информационным службам;
	\item методы обхода безопасности, например, программы, которые используют ошибки в операционных системах и прикладном программном обеспечении;
	\item захват открытых сетевых подключений.
\end{itemize}

\subsection{Общестатистическая информация по искусственным нарушениям информационной безопасности}
\label{subsec:problems:stats}

При просмотре любого списка статистических данных по кибербезопасности всегда важно учитывать следующее: данные будут отличаться в зависимости от источника, и сообщается не обо всех киберинцидентах и киберпреступлениях.
Различные организации используют разные квалификаторы и методологии в своей отчетности с точки зрения того, что может квалифицироваться как киберинцидент или утечка данных.

Кроме того, исследование обычно основывается на данных их собственных внутренних систем, данных мониторинга клиентов или информации, сообщенной жертвами киберпреступлений, или на опросах людей из определенных отраслей. И учитывая, что на обнаружение некоторых нарушений или
кибератак могут потребоваться недели, месяцы или даже годы – если они
вообще обнаружены – это означает, что фактические цифры могут быть выше
(или ниже), чем сообщается.

Это лишь некоторые из причин, по которым разные компании и
источники показывают разную информацию. Имея это в виду, можно
взглянуть на основную статистику за 2019-2020 годы:

Согласно отчету IC3 \cite{bardaev} (Internet Crime Complaint Center’s) о
преступности в Интернете за 2019 год, более \$ 1,7 млрд было утеряно в
результате взлома корпоративной электронной почти и мошенничества со
взломом учётной записи электронной почты (23 775 жалоб).

В 2019 году в результате киберпреступлений во всем мире было
потеряно более 3,5 миллиарда долларов. В отчете IC3 за этот период
говорится, что в общей сложности 467 351 инцидент был зарегистрирован
компаниями и частными лицами.

\subsection{Оценка потенциального ущерба от реализации угрозы}
\label{subsec:problems:damage_assessment}

При оценке риска должны учитываться как вероятность, так и серьезность риска нарушения, или свобода субъектов, которые имеют доступ к данных. Также было установлено, что риск нарушения должен
оцениваться на основе объективной оценки.
Следует подчеркнуть, что при оценке риска нарушения прав и свобод
физических лиц в результате нарушения акцент делается на других вопросах,
помимо вопросов риска, включенных в DPIA (Data Protection Impact
Assessment – процедура оценки влияния мер для защиты персональных
данных). DPIA \cite{bardaev} учитывает как риски для запланированной обработки
данных, так и риски, возникающие в случае нарушения. При изучении
возможного нарушения обычно учитывается вероятность его возникновения и
вред, который оно может причинить субъектам данных; Другими словами, это
оценка гипотетического события. В случае фактического нарушения, событие
уже произошло, поэтому основное внимание уделяется возникающему риску
того, что нарушение повлияет на людей.
DPIA – это процесс, результатом которого является оценка того,
существует ли риск при обработке персональных данных. Примерами
нарушений являются кража личных данных, дискриминация, финансовые
потери, ущерб публичности и нарушение профессиональной тайны. DPIA
проводится для таких процессов обработки данных, как набор персонала,
рассылка новостей и мониторинг. Оценка охватывает все операции в данном
процессе – от момента получения данных до их архивирования или уничтожения.
